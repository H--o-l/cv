\documentclass[10pt,a4paper]{moderncv}
\moderncvtheme[blue]{classic}
\usepackage[utf8]{inputenc}
\usepackage[scale=0.8]{geometry}


\firstname{Hoel}
\familyname{IRIS}
\title{\large Ingénieur Informatique Systèmes Embarqués}

\extrainfo{Né le 22 Juillet 1989}
\address{54 avenue Mathurin Moreau}{75019 Paris}
\mobile{0618393011}
\email{hoel.iris@gmail.com}
\photo[64pt]{photo_reduite}

\setlength{\hintscolumnwidth}{3cm}
\setlength{\topmargin}{-75pt}
\setlength{\textheight}{704pt}



\begin{document}

\maketitle

\section{Expériences}

  \cventry{Depuis\\Décembre 2015}
  {Ingénieur intégration continue}
  {ALSTOM en mission pour ClearSy}
  {}
  {Paris}
  {Intégration continue pour le métro Urbalis Fluence:\newline{}
  - Gestion de source et de configuration (multiples depôts) via Git et Gradle.\newline{}
  - Mise en place de Gitlab pour le suivi de modification et d'intégration.\newline{}
  - Développement des chaines de compilation redondées pour logiciels SIL4 sur gradle.\newline{}
  - Mise en place de Jenkins pipeline pour compilations de nuit sur un parc de machines.\newline{}}

  \cventry{Novembre 2012 à\\Décembre 2015}
  {Ingénieur développement logiciel}
  {ALSTOM en mission pour ClearSy}
  {}
  {Paris}
  {Développement du logiciel de conduite automatique des métros Urbalis Evolution:\newline{}
  - Animation des réunions hebdomadaires avec les equipes systèmes.\newline{}
  - Spécifications des exigences et de l'architecture du système.\newline{}
  - Développement embarqué en C du logiciel de conduite automatique.\newline{}
  - Interface avec les équipes de validation et analyse des retours.\newline{}}

  \cventry{Février à\\juillet 2012}
  {Projet de fin d'étude}
  {CEA}
  {}
  {Grenoble}
  {Etude et développement de plateformes de capteurs sans-fil pour le déploiement d'un middleware de haut niveau.\newline{}}
  \cventry{Juin à\\septembre 2011}
  {Stage}
  {ST Microelectronic}
  {}
  {Grenoble}
  {Portage du système d'exploitation Contiki et optimisation basse consommation pour le développement d'un réseau de capteurs.\newline{}}
  
\section{Compétences}

  \subsection{Informatique}

    \cvcomputer{Maîtrise}
    {\small Développement et intégration de logiciels embarqués.\newline{}}
    {Langages}
    {\small C, Python, Gradle, Groovy}
    \cvcomputer{Environnement\\ de développement}
    {\small Visual studio, Linux, Git, GitLab, Jenkins\newline{}}
    {Autre}
    {\small Ferroviaire, Réseaux de capteurs, Raspberry, Arduino, \LaTeX} 

  \subsection{Langues}

    \cvline{Anglais}
    {\small Ecrit courant, parlé occasionnel.\newline{}}

\section{Projets personnels}
  \cvline{Machine a café}
  {\small Intégration dans une machine à café d'un arduino avec une radio xbee, un relais et un capteur de température infrarouge.}
  
  \cvline{Domotique}
  {\small Pilotage d'ampoules wifi et de prises RF 433MHz via un serveur raspberry et un interrupteur arduino + radio nRF24.}
  
  \cvline{Alerte email}
  {\small Drapeau couplé a un servomoteur sur un raspberry pour prévenir de l'arrivée de nouveaux emails.\newline{}}
  
  \cvline{Associatif}
  {\small Coupe de France de robotique 2010 et 2011.}

\section{Diplômes et Études}
  \cventry{2012}
  {Ingénieur Grenoble INP - Phelma Ensimag}
  {}
  {}
  {}
  {Spécialité Systèmes et Logiciels Embarqués\newline{}}
  \cventry{2009}
  {Classes préparatoires}
  {}
  {}
  {Chartres}
  {Spécialité Physique Chimie\newline{}}

\section{Centres d'intérêt}
  \cvline{Autre}
  {\small Aéronautique, licence de pilote privé en 2008.}
  \cvline{}
  {\small Reportages scientifiques.}
  \cvline{}
  {\small Ancien élu au Conseil d'Administration de Grenoble INP.}

\end{document}
