\documentclass[10pt,a4paper]{moderncv}
\moderncvtheme[blue]{classic}
\usepackage[utf8]{inputenc}
\usepackage[scale=0.8]{geometry}


\firstname{Hoel}
\familyname{IRIS}
\title{\large Ingénieur Informatique Systèmes Embarqués}

\extrainfo{Né le 22 Juillet 1989}
\address{54 avenue Mathurin Moreau}{75019 Paris}
\mobile{0618393011}
\email{hoel.iris@gmail.com}
\photo[64pt]{photo_reduite}

\setlength{\hintscolumnwidth}{3cm}
\setlength{\topmargin}{-45pt}
\setlength{\textheight}{704pt}



\begin{document}

\maketitle

\section{Expériences}

  \cventry{Depuis\\Novembre 2015}
  {Ingénieur intégration logiciel}
  {ALSTOM en mission pour ClearSy}
  {}
  {Paris}
  {Intégration continue pour le métro Urbalis Fluence:\newline{}
  - Gestion de source et de configuration. \newline{}
  - Intégration logicielle et suivi de modification de code. \newline{}
  - Mise en place de Gitlab-CI pour évaluer les requêtes d'intégrations.\newline{} 
  - Automatisation des chaines de traduction de code, de compilation et de livraison.\newline{} 
  - Développement des outils d'intégration continue: compilation, test et suivi de qualité quotidiens.\newline{} }
  
  \cventry{Novembre 2012 à\\Novembre 2015}
  {Ingénieur développement logiciel}
  {ALSTOM en mission pour ClearSy}
  {}
  {Paris}
  {Développement logiciel pour les métros Urbalis evolution:\newline{}
  - Animation des réunions hebdomadaires avec les équipes systèmes.\newline{}
  - Spécifications des exigences et de l’architecture du système.\newline{}
  - Développement embarqué du logiciel de conduite automatique.\newline{}
  - Interface avec les équipes de validation et analyse des retours.\newline{}}

  \cventry{Février à\\juillet 2012}
  {Projet de fin d'étude}
  {CEA}
  {}
  {Grenoble}
  {Prototypage de capteurs sans-fil, implémentations des primitives d'un middleware et étude des performances.\newline{}}

\section{Diplômes et Études}
  \cventry{2012}
  {Ingénieur Grenoble INP - Phelma Ensimag}
  {}
  {}
  {}
  {Spécialité Systèmes et Logiciels Embarqués\newline{}}
  \cventry{2009}
  {Classes préparatoires}
  {}
  {}
  {Chartres}
  {Spécialité Physique Chimie\newline{}}

\section{Compétences}

  \subsection{Informatique}

    \cvcomputer{Maîtrise}
    {Développement et intégration de logiciels embarqués.\newline{}}
    {Langages}
    {C, Python, Gradle, Groovy}
    \cvcomputer{Environnement\\ de développement}
    {Visual studio, Linux, Git, GitLab, Jenkins\newline{}}
    {Autre}
    {Ferroviaire, Réseaux de capteurs, Raspberry, Arduino, \LaTeX} 

  \subsection{Langues}

    \cvline{Anglais}
    {\small Ecrit courant, parlé occasionnel.\newline{}}

  \section{Centres d'intérêt}
      \cvline{Projets personnels}
      {\small Machine à café, domotique et alerte email.\newline{}
              GitHub: https://github.com/H--o-l?tab=repositories\newline{}}

      \cvline{Associatif}
      {\small Coupe de France de robotique 2010 et 2011.\newline{}
              Licence de pilote privé en 2008.}

\end{document}
